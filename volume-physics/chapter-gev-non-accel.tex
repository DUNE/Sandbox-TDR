\chapter{GeV-scale non-accelerator physics program}
\label{ch:nonaccel}


%%%%%%%%%%%%%%%%%%%%%%%%%%%%%%%%%%%%%%%%%%%%%%%%%%%%%%%%%%%%%%%%
\section{Reference LaTeX code}

Please reference the guidance document at 

\begin{itemize}
\item Bare LaTeX files: https://github.com/DUNE/document-guidance
\item Compiled version: ``guidance.pdf'' at https://dune.bnl.gov/docs/
\end{itemize}

\fixme{Remove this section when you're done editing or you no longer need it.}

%%%%%%%%%%%%%%%%%%%%%%%%%%%%%%%%%%%%%%%%%%%%%%%%%%%%%%%%%%%%%%%%
\section{Nucleon decay}
\label{sec:nonaccel-ndk}

\subsection{Predictions from Grand Unified Theories and current experimental status}
\label{subsec:nonaccel-ndk-status}

\begin{description}
\item[Predictions and experimental status:] Plot or table with partial lifetime predictions for dominant NDK channels in selected GUT theories, together with current limits for each channel.
\end{description}

\subsection{Experimental signatures for nucleon decay searches in DUNE}
\label{subsec:nonaccel-ndk-dune}

\begin{description}
\item[Impact of nuclear effects:] Plot showing MC truth multiplicity and kinetic energy distributions of particles exiting Ar nucleus for selected nucleon decay modes. Before/after final state interactions in Ar nucleus.
\item[Relevant backgrounds:] Plot or table illustrating event rates for atmospheric neutrino and cosmic ray muon interactions in the DUNE FD and surrounding material.
\item[Experimental signature:] Event display of reconstructed 3D hits in LAr for selected nucleon decay events.
\item[Event reconstruction and classification performance:] plots illustrating DUNE-specific strengths for NDK searches. Important quantities to show for selected NDK modes may include: track/shower reconstruction efficiency as a function of KE, invariant mass resolution, spatial resolution for displaced vertices, $e/\gamma/\mu/\pi/K/p$ separation, example of DNN-based classification performance.  
\end{description}

\subsection{Sensitivity to $p\to\bar{\nu}K^+$ decay}
\label{subsec:nonaccel-ndk-nubarkplus}

\begin{description}
\item[Selection:] Plot or table illustrating $p\to\bar{\nu}K^+$ event selection, and corresponding signal efficiency and background rates.
\item[Sensitivity:] Plot of $p\to\bar{\nu}K^+$ partial lifetime sensitivity versus exposure, or versus years.
\end{description}

\subsection{Sensitivity to other key nucleon decay modes}
\label{subsec:nonaccel-ndk-other}

\begin{description}
\item[Sensitivity:] Plot or table summarising signal efficiency, background rate, and sensitivity at fixed exposure for other selected nucleon decay modes. Important other modes to highlight: proton to charged lepton plus neutral kaon, proton to charged lepton plus unflavored neutral meson.
\end{description}


\subsection{Detector requirements for nucleon decay searches}
\label{subsec:nonaccel-ndk-requirements}

\begin{description}
\item[Trigger efficiency:] Plot or table illustrating nucleon decay event trigger efficiency as a function of efficiency/coverage of photon detection system.
\end{description}


%%%%%%%%%%%%%%%%%%%%%%%%%%%%%%%%%%%%%%%%%%%%%%%%%%%%%%%%%%%%%%%%
\section{N-Nbar oscillations}
\label{sec:nonaccel-nnbar}

\subsection{Motivation for $\Delta$B=2 physics and possible experimental approaches}
\label{subsec:nonaccel-nnbar-intro}

\begin{description}
\item[Annihilation modes:] Plot or table giving branching ratios for antineutron annihilation modes applicable to intranuclear searches.
\end{description}


\subsection{Sensitivity to intranuclear neutron-antineutron oscillations in DUNE}
\label{subsec:nonaccel-nnbar-dunesensitivity}

\begin{description}
\item[Impact of nuclear effects:] Plot showing MC truth of total event kinetic energy and/or momentum distributions. Before/after final state interactions in Ar nucleus.
\item[Experimental signature:] Event display of reconstructed 3D hits in LAr for a selected n-nbar event and an atmospheric neutrino background event.
\item[Selection:] Plot or table illustrating n-nbar event selection, and corresponding signal efficiency and background rates.
\item[Sensitivity:] Plot of n-nbar oscillation lifetime sensitivity versus exposure, or versus years.
\end{description}


%%%%%%%%%%%%%%%%%%%%%%%%%%%%%%%%%%%%%%%%%%%%%%%%%%%%%%%%%%%%%%%%
\section{Physics with atmospheric neutrinos}
\label{sec:nonaccel-atm}

\subsection{Oscillation physics with atmospheric neutrinos}
\label{sec:nonaccel-atm-oscillations}

\subsection{BSM physics with atmospheric neutrinos}
\label{sec:nonaccel-atm-bsm}

\subsection{Reconstruction of atmospheric neutrinos}
\label{sec:nonaccel-atm-reco}
