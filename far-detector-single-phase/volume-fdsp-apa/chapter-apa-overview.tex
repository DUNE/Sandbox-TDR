\chapter{Overview of Anode Plane Assemblies}
\label{ch:fdsp-apa-ov}

\fixme{Some initial text suggested by Anne (in fact the APA section in the protodune SP TDR has a lot of good descriptive text)}

%%%%%%%%%%%%%%%%%%%%%%%%%%%%%%%%%%%%%%%%%%%%%%%%%%%%%%%%%%%%%%%%%%%%
\section{Introduction}
\label{sec:fdsp-apa-intro}

Anode Plane Assemblies (APAs) are the far detector elements utilized to sense ionization created by
charged particles traversing the liquid argon volume inside the single-phase TPC. The planes are interleaved with Cathode Plane Assemblies (CPAs), as shown in Figure..., to establish the required electric fields and form drift volumes for the charged particles. 

\fixme{Include an image of the overall system, indicating its parts. Show how the system fits into the overall detector.}

The operating principle is illustrated in Figure... (add figure)

An APA consists of a rectangular framework with a fine wire mesh stretched across it, over which are wrapped four layers of sense and shielding wires...
...

%%%%%%%%%%%%%%%%%%%%%%%%%%%%%%%%%%%%%%%%%%%%%%%%%%%%%%%%%%%%%%%%%%%%
\section{Design Considerations}
\label{sec:fdsp-apa-des-consid}


%%%%%  Design to identify MIPs -- wire pitch and other params
The APA design must enable identification of minimum-ionizing particles (MIPs). This is a function of several detector parameters, including: argon purity, drift distance, diffusion, wire pitch, and Equivalent Noise Charge (ENC).  DUNE-SP requires that MIPs originating anywhere inside the active volume of the detector be reconstructed with 100$\%$ efficiency.   The choice of wire pitch, of $\sim$5\,mm, for the wire layers on the APA, combined with key parameters for other TPC systems (described in their respective sections of the TDR), is expected to enable the 100$\%$ MIP identification efficiency.

%%%%% Locate vertices, determine fiducial vol, then back to vertex...?
DUNE-SP requires that it be possible to determine the fiducial volume (via analysis) to $<$1$\%$, which in turn requires reaching a vertex resolution of $\sim$1.5\,cm along each coordinate direction. (The fiducial volume, among other factors, determines the number of target nucleons, which is a component in cross section measurements.) 
The fine granularity of the APA wires enables excellent precision in identifying the location of any vertices in an event, (e.g., the primary vertex in a neutrino interaction or gamma conversion points in a $\pi^{0}$ decay), which has a direct impact on reconstruction efficiency.
In practice, the resolution on the drift-coordinate ($x$) of a vertex or hit will be better than that on its location in the $y-z$ plane, due to the combination of drift-velocity and electronics sampling-rate...

...

\fixme{Anne suggests: Within this section add ref to requirements document  when it's ready, and maybe list the most important half dozen in a table here). E.g.,}  

\begin{dunetable}
[Important requirements on the APA design]
{p{0.8\textwidth}}
{apaphysicsparams}
{Important requirements on the APA design (Sample only!)}   
Requirement  \\ \toprowrule
 The APA wire spacing shall be chosen to provide high efficiency in distinguishing electron shower from photon showers \\ \colhline
 The APA wire planes shall be oriented relative to each other to optimize the measurement of high energy and low energy tracks from accelerator  \\ \colhline
 ...\\ 
\end{dunetable}

\fixme{By the end of the volume, for every requirement listed in this section, there should exist an explanation of how it will be satisfied.}


%%%%%%%%%%%%%%%%%%%%%%%%%%%%%%%%%%%%%%%%%%%%%%%%%%%%%%%%%%%%%%%%%%%%
\section{Scope}
\label{sec:fdsp-apa-scope}

The scope of the Anode Plane Assembly (APA) system includes the continued procurement of materials for, and the fabrication, testing, delivery and installation of the following systems: 

\begin{itemize}
\item stainless steel APA frame, 150 per module 
\item wire mesh that covers the frame on both sides, ...
\end{itemize}